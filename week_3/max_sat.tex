% Created 2024-06-16 Sun 18:37
% Intended LaTeX compiler: pdflatex
\documentclass[11pt]{article}
\usepackage[utf8]{inputenc}
\usepackage[T1]{fontenc}
\usepackage{graphicx}
\usepackage{longtable}
\usepackage{wrapfig}
\usepackage{rotating}
\usepackage[normalem]{ulem}
\usepackage{amsmath}
\usepackage{amssymb}
\usepackage{capt-of}
\usepackage{hyperref}
\usepackage[a4paper,left=1cm,right=1cm,top=1cm,bottom=1cm]{geometry}
\usepackage[T1]{fontenc}
\usepackage[utf8]{inputenc}
\usepackage[american]{babel}
\usepackage[sc]{mathpazo}
\linespread{1.05}
\author{Marcio Woitek}
\date{}
\title{Max-SAT Approximation}
\hypersetup{
 pdfauthor={Marcio Woitek},
 pdftitle={Max-SAT Approximation},
 pdfkeywords={},
 pdfsubject={},
 pdfcreator={Emacs 29.3 (Org mode 9.6.24)}, 
 pdflang={English}}
\begin{document}

\maketitle

\section*{Problem 1}
\label{sec:orgffad40c}
\begin{equation}
4 \cdot \frac{7}{8} = \frac{7}{2} = 3.5
\end{equation}

\section*{Problem 2}
\label{sec:orgd521fc5}
There is a truth assignment that satisfies all the clauses.

\section*{Problem 3}
\label{sec:orgd9b2d24}
By setting \(x_1 = \mathrm{T}\), the clauses involving \(x_1\) reduce to
\begin{eqnarray*}
x_1 \vee x_2 \vee \overline{x}_4 &=& \mathrm{T} \\
x_1 \vee \overline{x}_2 \vee x_3 &=& \mathrm{T}.
\end{eqnarray*}
The other 2 clauses remain unchanged. Then the expected number of clauses we can
satisfy is given by
\begin{equation}
1 + \frac{7}{8} + 1 + \frac{7}{8} = 3 + \frac{3}{4} = 3.75
\end{equation}

\section*{Problem 4}
\label{sec:org8e27c71}
By setting \(x_1 = \mathrm{F}\), the clauses involving \(x_1\) reduce to
\begin{eqnarray*}
x_1 \vee x_2 \vee \overline{x}_4 &=& x_2 \vee \overline{x}_4 \\
x_1 \vee \overline{x}_2 \vee x_3 &=& \overline{x}_2 \vee x_3.
\end{eqnarray*}
The other 2 clauses remain unchanged. Then the expected number of clauses we can
satisfy is given by
\begin{equation}
\frac{3}{4} + \frac{7}{8} + \frac{3}{4} + \frac{7}{8} = \frac{13}{4} = 3.25
\end{equation}

\section*{Problem 5}
\label{sec:org0249b24}
True
\end{document}
